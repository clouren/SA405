% LaTeX Article Template
\documentclass[12pt]{article}
%% Other packages
\usepackage{amsmath}
\usepackage{amsthm}
\usepackage{titlesec}
\usepackage{soul}
\usepackage{tikz}
\usepackage{tikz-3dplot}
\usepackage{amssymb}
\usepackage{multicol}
\usepackage{float}
\usepackage{calc}
\usepackage{fancybox}
\usepackage{array}
\usepackage[shortlabels]{enumitem}
\usepackage{framed}
\usepackage{hyperref}
\newcolumntype{L}[1]{>{\raggedright\let\newline\\\arraybackslash\hspace{0pt}}m{#1}}
\newcolumntype{C}[1]{>{\centering\let\newline\\\arraybackslash\hspace{0pt}}m{#1}}
\newcolumntype{R}[1]{>{\raggedleft\let\newline\\\arraybackslash\hspace{0pt}}m{#1}}


%% Margins
\usepackage{geometry}
\geometry{verbose,letterpaper,tmargin=1in,bmargin=1in,lmargin=1in,rmargin=1in}

\newcommand{\menuchoice}[2]{{\ttfamily#1..#2}}
\newcommand{\dotdot}{..}

\usepackage{graphicx}

% Array vertical and horizontal stretch
% \def\arraystretch{1.5}%  1 is the default, change whatever you need
% \setlength{\tabcolsep}{12pt}

%\graphicspath{%
\graphicspath{{./figs/}}

%% Paragraph style settings
\setlength{\parskip}{\medskipamount}
\setlength{\parindent}{0pt}

%% Change itemize bullets
\renewcommand{\labelitemi}{$\bullet$}
\renewcommand{\labelitemii}{$\circ$}
\renewcommand{\labelitemiii}{$\diamond$}
\renewcommand{\labelitemiv}{$\cdot$}

%% Shrink section fonts
\titleformat*{\section}{\large\bf}
\titleformat*{\subsection}{\normalsize\it}
\titleformat*{\subsubsection}{\normalsize\bf}

% %% Compress the spacing around section titles
\titlespacing*{\section}{0pt}{1.5ex}{0.75ex}
\titlespacing*{\subsection}{0pt}{1ex}{0.5ex}
\titlespacing*{\subsubsection}{0pt}{1ex}{0.5ex}

%% amsthm settings
\theoremstyle{definition}
\newtheorem{problem}{Problem}
\newtheorem{example}{Example}
\newtheorem{mydef}{Definition}

%% Answer box macros
%% \answerbox{alignment}{width}{height}
\newcommand{\answerbox}[3]{%
  \fbox{%
    \begin{minipage}[#1]{#2}
      \hfill\vspace{#3}
    \end{minipage}
  }
}

%% \answerboxfull{alignment}{height}
\newcommand{\answerboxfull}[2]{%
  \answerbox{#1}{\textwidth}{#2} 
}

%% \answerboxone{alignment}{height} -- for first-level bullet
\newcommand{\answerboxone}[2]{%
  \answerbox{#1}{6.15in}{#2} 
}

%% \answerboxtwo{alignment}{height} -- for second-level bullet
\newcommand{\answerboxtwo}[2]{%
  \answerbox{#1}{5.8in}{#2}
}

%% \graphbox{xmin}{xmax}{ymin}{ymax}{scale}
\newcommand{\graphbox}[5]%[-5, 5, -5, 5, 0.33]
{
\begin{tikzpicture}
     [>=latex,scale=#5]
     
     % Coordinate axes
     \draw [->,very thick] (#1, 0) -- (#2, 0) node[right] {$x$};
     \draw [->,very thick] (0, #3) -- (0, #4) node[above] {$y$};
     
     % Grid
     \draw[step=1cm,thick,dotted] (#1,#3) grid (#2,#4);
   \end{tikzpicture}
   }


%% Redefine maketitle
\makeatletter
\renewcommand{\maketitle}{
  \noindent SA405 -- AMP \hfill  Reading: \S 2.1 \\

  \begin{center}\Large{\textbf{\@title}}\end{center}
}
\makeatother

% Set the beginning of a LaTeX document
\begin{document}

%\graphbox{-10}{3}{-5}{10}

\title{Lesson 1: Linear Programming Review, Intro to Integer Programming}

%\graphbox[10][10]

\maketitle

\section{Goals}
\begin{itemize}
\item  Formulate a concrete linear programming model.
\item  Introduce an integrality requirement on variables.
\item  Convert the integer program to parameterized form.
\end{itemize}

\section{Review of SA305 Formulations}

The five components of formulating an optimization model are:

\begin{enumerate}
\item \phantom{a} 
\item \phantom{a} \vspace{0.75in}
\item \phantom{a} \vspace{0.75in}
\item \phantom{a} \vspace{0.75in}
\item \phantom{a} \vspace{0.75in}
\end{enumerate}

\newpage

What are the 3 assumptions/characteristics that make an optimization model a linear program?
\begin{enumerate}
\item \phantom{a}
\item \phantom{a}
\item \phantom{a}
\end{enumerate}


\section{Concrete Model}

Chelsea is heading out on a camping trip, and she wants to carry only one pack that has 5.3 ft$^3$ of volumetric space. To keep from hurting her back, she needs to make sure that the contents of her backpack weighs no more than 12.5 lbs. You can assume the backpack weight is negligible. See the list of items that she is able to bring:

\begin{center}
\begin{tabular}{|c|c |c| c| c|}
\hline
~ID~ & Item & Volume (ft$^3$) & Usefulness Factor & Weight (lbs.)\\ \hline
1 & Rope & 2 & 1 & 3 \\ \hline
2 & Matches & 0.01 & 5 & 0.1 \\  \hline
3 & Tent & 3 & 7 & 10   \\ \hline
4 & Sleeping bag & 2 & 6 & 4   \\ \hline
5 & Hammock & 0.4 & 4.5 & 4   \\ \hline
6 & Granola bars & 0.67 & 8 & 2   \\ \hline
\end{tabular}
\end{center}

This problem is referred to as the \textbf{knapsack} problem and is a very widely used type of integer program. We'll see why we need IP shortly. We will first formulate it as an LP.


\begin{problem}  Write a concrete linear program whose solution maximizes the usefulness of the contents of Chelsea's bag given volume and weight requirements.

\begin{enumerate}[a)]
\item Define decision variables and then describe the objective function and the role of each constraint. 

\pagebreak

\item Write the concrete model.

%\textbf{\underline{Parameters}} \vspace{1in}

%\textbf{\underline{Decision Variables}} \vspace{1.5in}

%\textbf{\underline{Objective Function}} \vspace{1.5in}

%\textbf{\underline{Constraints}}


\end{enumerate}
\end{problem}

\vfill
\newpage

\section{Integrality Restrictions}

Suppose we solve this LP and get the following solution:
\[
x_{granola} = x_{hammock} = x_{matches} = x_{sleeping} = 1
\]
\[
x_{tent} = 0.24
\]
\[
x_{rope} = 0
\]

What is the objective function value of this solution? Denote this solution as $z^*_{LP}$. \vfill

Is this a reasonable solution? \vfill

Suppose we change the problem so that $x_i$ is \textbf{binary} instead of continuous. A binary variable is only allowed to take the value of 0 or 1. That is, the variables would be changed to: \vfill

Our problem has now become a \textbf{integer program}. Why? \vfill

Let $z^*_{IP}$ be the optimal objective function value of the integer program. Would we expect $z^*_{IP}$ to be smaller, larger, or the same as $z^*_{LP}$?

\vfill

\pagebreak
\section{Convert to Parameterized Models}
\begin{problem}
Assuming integrality restrictions, convert your model to a parameterized model.  Clearly define all sets, parameters, and decision variables. \\
\end{problem}


\end{document}
