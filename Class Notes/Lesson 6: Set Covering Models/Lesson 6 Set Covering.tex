\documentclass[11pt]{article}

%% MinionPro fonts 
%\usepackage[lf]{MinionPro}
%\usepackage{MnSymbol}
\usepackage{microtype}

%% Margins
\usepackage{geometry}
\geometry{verbose,letterpaper,tmargin=1in,bmargin=1in,lmargin=1in,rmargin=1in}

%% Other packages
\usepackage{amsmath}
\usepackage{amsthm}
\usepackage{amsfonts}
\usepackage{mathrsfs}
\usepackage[shortlabels]{enumitem}
\usepackage{titlesec}
\usepackage{soul}
\usepackage{tikz}
\usepackage{mathtools}
\usepackage{pgfplots}
\usepackage{tikz-3dplot}
\usepackage{algorithmic}
\usepackage[export]{adjustbox}
\usepackage{tcolorbox}

%% Paragraph style settings
\setlength{\parskip}{\medskipamount}
\setlength{\parindent}{0pt}

%% Change itemize bullets
\renewcommand{\labelitemi}{$\bullet$}
\renewcommand{\labelitemii}{$\circ$}
\renewcommand{\labelitemiii}{$\diamond$}
\renewcommand{\labelitemiv}{$\cdot$}

%% Colors
\definecolor{rred}{RGB}{204,0,0}
\definecolor{ggreen}{RGB}{0,145,0}
\definecolor{yyellow}{RGB}{255,185,0}
\definecolor{bblue}{rgb}{0.2,0.2,0.7}
\definecolor{ggray}{RGB}{190,190,190}
\definecolor{ppurple}{RGB}{160,32,240}
\definecolor{oorange}{RGB}{255,165,0}

%% Shrink section fonts
\titleformat*{\section}{\normalsize\bf}
\titleformat*{\subsection}{\normalsize\bf}
\titleformat*{\subsubsection}{\normalsize\it}

% %% Compress the spacing around section titles
\titlespacing*{\section}{0pt}{1.5ex}{0.75ex}
\titlespacing*{\subsection}{0pt}{1ex}{0.5ex}
\titlespacing*{\subsubsection}{0pt}{1ex}{0.5ex}

%% amsthm settings
\theoremstyle{definition}
\newtheorem{problem}{Problem}
\newtheorem{example}{Example}
\newtheorem*{theorem}{Theorem}
\newtheorem*{bigthm}{Big Theorem}
\newtheorem*{biggerthm}{Bigger Theorem}
\newtheorem*{bigcor1}{Big Corollary 1}
\newtheorem*{bigcor2}{Big Corollary 2}

%% tikz settings
\usetikzlibrary{calc}
\usetikzlibrary{patterns}
\usetikzlibrary{decorations}
\usepgfplotslibrary{polar}

%% algorithmic setup
\algsetup{linenodelimiter=}
\renewcommand{\algorithmiccomment}[1]{\quad// #1}
\renewcommand{\algorithmicrequire}{\emph{Input:}}
\renewcommand{\algorithmicensure}{\emph{Output:}}

%% Answer box macros
%% \answerbox{alignment}{width}{height}
\newcommand{\answerbox}[3]{%
  \fbox{%
    \begin{minipage}[#1]{#2}
      \hfill\vspace{#3}
    \end{minipage}
  }
}

%% \answerboxfull{alignment}{height}
\newcommand{\answerboxfull}[2]{%
  \answerbox{#1}{6.38in}{#2} 
}

%% \answerboxone{alignment}{height} -- for first-level bullet
\newcommand{\answerboxone}[2]{%
  \answerbox{#1}{6.0in}{#2} 
}

%% \answerboxtwo{alignment}{height} -- for second-level bullet
\newcommand{\answerboxtwo}[2]{%
  \answerbox{#1}{5.8in}{#2}
}

%% special boxes
\newcommand{\wordbox}{\answerbox{c}{1.2in}{.5cm}}
\newcommand{\catbox}{\answerbox{c}{.5in}{.7cm}}
\newcommand{\letterbox}{\answerbox{c}{.7cm}{.5cm}}

%% Miscellaneous macros
\newcommand{\tstack}[1]{\begin{multlined}[t] #1 \end{multlined}}
\newcommand{\cstack}[1]{\begin{multlined}[c] #1 \end{multlined}}
\newcommand{\ccite}[1]{\only<presentation>{{\scriptsize\color{gray} #1}}\only<article>{{\small [#1]}}}
\newcommand{\grad}{\nabla}
\newcommand{\ra}{\ensuremath{\rightarrow}~}
\newcommand{\maximize}{\text{maximize}}
\newcommand{\minimize}{\text{minimize}}
\newcommand{\subjectto}{\text{subject to}}
\newcommand{\trans}{\mathsf{T}}
\newcommand{\bb}{\mathbf{b}}
\newcommand{\bx}{\mathbf{x}}
\newcommand{\bc}{\mathbf{c}}
\newcommand{\bd}{\mathbf{d}}

%% LP format
%    \begin{align*}
%      \maximize \quad & \mathbf{c}^{\trans} \mathbf{x}\\
%      \subjectto \quad & A \mathbf{x} = \mathbf{b}\\
%                       & \mathbf{x} \ge \mathbf{0}
%    \end{align*}


%% Redefine maketitle
\makeatletter
\renewcommand{\maketitle}{
  \noindent SA405 -- AMP \hfill Rader \S 3.2 \\

  \begin{center}\Large{\textbf{\@title}}\end{center}
}
\makeatother

%% ----- Begin document ----- %%
\begin{document}
  
\title{Lesson 6: Set Covering, Packing, and Partitioning}

\maketitle

%%%

\section{Covering Students}
The USNA would like all plebes to hear a presentation about major selection. An OR student, Hannah, decides to visit some of the students in order to tell them about which major is the absolute best. She wants to make sure that each plebe sees the presentation, but would like to visit as few classes as possible.  She develops the following mini-version of the problem in order to help write a model that will solve the large-scale optimization problem.

Let $S$ be the set of students:

~~~$S := \{$ Kyle, Aaron, Ryan, Jordan, Monika, Brandon, Sharon, Adam, Natalie, Joshua $\}$

Let $\mathscr{C}$ be the set of classes:

~~~$\mathscr{C} := \{$ Naval history, Fencing, Sailing, Boxing, Wrestling, Calculus $\}$

~~~Each element $C$ of $\mathscr{C}$ is itself a set, a subset of $S$ ($C \subseteq S, \text{ for all } C \in \mathscr{C}$):
\begin{center}
\begin{tabular}{ll}
{\bf N}aval history &$:= \{$ Kyle, Ryan, Monika, Brandon $\}$\\
{\bf F}encing &$:= \{$ Kyle, Jordan, Samnang, Natalie $\}$\\
{\bf S}ailing &$:= \{$ Aaron, Monika, Adam $\}$\\
{\bf B}oxing &$:= \{$ Aaron, Ryan, Jordan, Samnang $\}$\\
{\bf W}restling &$:= \{$ Jordan, Brandon, Joshua $\}$\\
{\bf C}alculus &$:= \{$ Adam, Natalie, Joshua $\}$\\
\end{tabular}
\end{center}

\bigskip
Hannah defines the following set of binary variables:

\[
z_C := \left\{ \begin{array}{ll}
1 & \text{ if she should visit class $i$ } \\
0 & \text{ if she should not visit class $i$ } \\
\end{array} \right. \text{, for $C \in \mathscr{C}$}
\]

%%%
\newpage
\section{Set Covering}

\begin{enumerate}

\item Write two concrete constraints:  one that ensures that Jordan will see the presentation, and one that ensures that Brandon will see the presentation.

\vspace{3in}

\item Why are these called {\bf set covering constraints}? (Think of the set of students.) \vspace{0.5in}

\item How many set covering constraints are needed? \vspace{1in}

\item Using the same sets as above and the variable $z_c$, how would we write a general parameterized set covering constraint for the students?

\newpage

The parameterized constraint above works but is a bit messy. There's another way to parameterize it using what's called an \textbf{adjacency matrix}. The adjacency matrix is a matrix where the rows correspond to the classes and the columns correspond to the students.

\item Let the adjacency matrix be $a_{c,s}$ for all $c \in \mathscr{C}$ and all $s \in \mathscr{S}$ Illustrate this matrix. \vspace{2in}

\item Write the parameterized set covering constraints using the adjacency matrix. \vfill

\begin{tcolorbox}
Either approach works, it's really up to you when it comes to modeling.
\end{tcolorbox}



\item Write a condensed abstract model to find a set of classes that covers all students while requiring the fewest possible presentations using the sets, variables, and parameters defined above.  
\vfill
\vfill
\end{enumerate}


%%%
\newpage
\section{Set Packing}
Eventually Hannah realizes that no student can stand to hear the presentation multiple times, but that she really wants lots of practice with public speaking.  She wants to give the presentation as many times as possible without any student seeing it more than once.
\begin{enumerate}

\item Write two concrete constraints:  one that ensures that Ryan will see the presentation \emph{at most once}, and one that ensures that Brandon will see the presentation \emph{at most once}.

\vspace{2in}

\item Why are these called {\bf set packing constraints}?  (Think of the set of classes.)
\vspace{0.5in}
\item  Write a condensed abstract model to find a collection of classes that maximizes the number of classes Hannah visits, while not seeing any student more than once. 
\vfill
\vfill
\end{enumerate}

%%%
\newpage
\section{Set Partitioning}
Hannah receives a message of encouragement from the Chief of Staff and is told to be sure to show the presentation to \emph{every single student}.  But she still knows that no student can possibly sit through it twice, so she must revise her model again.  
\begin{enumerate}
\item Write two concrete constraints:  one that ensures that Aaron will see the presentation \emph{exactly once}, and one that ensures that Sharon will see the presentation \emph{exactly once}.

\vspace{2in}

\item Why are these called {\bf set partitioning constraints}?  (Think of the set of students.) \vspace{0.5in}
\item  Write an abstract model to find a collection of classes that minimizes the number of classes Hannah visits, while seeing every student exactly once. 

\end{enumerate}






\end{document}
