\documentclass[11pt]{article}

%% MinionPro fonts 
%\usepackage[lf]{MinionPro}
%\usepackage{MnSymbol}
\usepackage{microtype}

%% Margins
\usepackage{geometry}
\geometry{verbose,letterpaper,tmargin=1in,bmargin=1in,lmargin=1in,rmargin=1in}

%% Other packages
\usepackage{amsmath}
\usepackage{amsthm}
\usepackage[shortlabels]{enumitem}
\usepackage{titlesec}
\usepackage{soul}
\usepackage{tikz}
\usepackage{mathtools}
\usepackage{pgfplots}
\usepackage{tikz-3dplot}
\usepackage{algorithmic}
\usepackage[export]{adjustbox}
\usepackage{tcolorbox}
\usepackage{optprog}

%% Paragraph style settings
\setlength{\parskip}{\medskipamount}
\setlength{\parindent}{0pt}

%% Change itemize bullets
\renewcommand{\labelitemi}{$\bullet$}
\renewcommand{\labelitemii}{$\circ$}
\renewcommand{\labelitemiii}{$\diamond$}
\renewcommand{\labelitemiv}{$\cdot$}

%% Colors
\definecolor{rred}{RGB}{204,0,0}
\definecolor{ggreen}{RGB}{0,145,0}
\definecolor{yyellow}{RGB}{255,185,0}
\definecolor{bblue}{rgb}{0.2,0.2,0.7}
\definecolor{ggray}{RGB}{190,190,190}
\definecolor{ppurple}{RGB}{160,32,240}
\definecolor{oorange}{RGB}{255,165,0}

%% Shrink section fonts
\titleformat*{\section}{\normalsize\bf}
\titleformat*{\subsection}{\normalsize\bf}
\titleformat*{\subsubsection}{\normalsize\it}

% %% Compress the spacing around section titles
\titlespacing*{\section}{0pt}{1.5ex}{0.75ex}
\titlespacing*{\subsection}{0pt}{1ex}{0.5ex}
\titlespacing*{\subsubsection}{0pt}{1ex}{0.5ex}

%% amsthm settings
\theoremstyle{definition}
\newtheorem{problem}{Problem}
\newtheorem{example}{Example}
\newtheorem*{theorem}{Theorem}
\newtheorem*{bigthm}{Big Theorem}
\newtheorem*{biggerthm}{Bigger Theorem}
\newtheorem*{bigcor1}{Big Corollary 1}
\newtheorem*{bigcor2}{Big Corollary 2}

%% tikz settings
\usetikzlibrary{calc}
\usetikzlibrary{patterns}
\usetikzlibrary{decorations}
\usepgfplotslibrary{polar}

%% algorithmic setup
\algsetup{linenodelimiter=}
\renewcommand{\algorithmiccomment}[1]{\quad// #1}
\renewcommand{\algorithmicrequire}{\emph{Input:}}
\renewcommand{\algorithmicensure}{\emph{Output:}}

%% Answer box macros
%% \answerbox{alignment}{width}{height}
\newcommand{\answerbox}[3]{%
  \fbox{%
    \begin{minipage}[#1]{#2}
      \hfill\vspace{#3}
    \end{minipage}
  }
}

%% \answerboxfull{alignment}{height}
\newcommand{\answerboxfull}[2]{%
  \answerbox{#1}{6.38in}{#2} 
}

%% \answerboxone{alignment}{height} -- for first-level bullet
\newcommand{\answerboxone}[2]{%
  \answerbox{#1}{6.0in}{#2} 
}

%% \answerboxtwo{alignment}{height} -- for second-level bullet
\newcommand{\answerboxtwo}[2]{%
  \answerbox{#1}{5.8in}{#2}
}

%% special boxes
\newcommand{\wordbox}{\answerbox{c}{1.2in}{.7cm}}
\newcommand{\catbox}{\answerbox{c}{.5in}{.7cm}}
\newcommand{\letterbox}{\answerbox{c}{.7cm}{.7cm}}

%% Miscellaneous macros
\newcommand{\tstack}[1]{\begin{multlined}[t] #1 \end{multlined}}
\newcommand{\cstack}[1]{\begin{multlined}[c] #1 \end{multlined}}
\newcommand{\ccite}[1]{\only<presentation>{{\scriptsize\color{gray} #1}}\only<article>{{\small [#1]}}}
\newcommand{\grad}{\nabla}
\newcommand{\ra}{\ensuremath{\rightarrow}~}
\newcommand{\maximize}{\text{maximize}}
\newcommand{\minimize}{\text{minimize}}
\newcommand{\subjectto}{\text{subject to}}
\newcommand{\trans}{\mathsf{T}}
\newcommand{\bb}{\mathbf{b}}
\newcommand{\bx}{\mathbf{x}}
\newcommand{\bc}{\mathbf{c}}
\newcommand{\bd}{\mathbf{d}}
\newcommand{\blu}{\color{blue}}

%% LP format
%    \begin{align*}
%      \maximize \quad & \mathbf{c}^{\trans} \mathbf{x}\\
%      \subjectto \quad & A \mathbf{x} = \mathbf{b}\\
%                       & \mathbf{x} \ge \mathbf{0}
%    \end{align*}


%% Redefine maketitle
\makeatletter
\renewcommand{\maketitle}{
  \noindent SA405 -- AMP \hfill Rader \#2.42 \\

  \begin{center}\Large{\textbf{\@title}}\end{center}
}
\makeatother

%% ----- Begin document ----- %%
\begin{document}
  
\title{HW8: IP Formulations}


\maketitle

\begin{enumerate}
\item Consider the following integer program.

\begin{optprog*}
max & \objective{2 x_1 + 3 x_2 - 4 x_3} \\
st  & x_1 + x_2 + 2x_3 & \leq & 7 \\
    & x_2 + x_3 & \geq & 1.25 \\
    & x_1 & \leq & 5 \\
    & x_1 & \geq & 0 \text{,integer} \\
    & x_2, x_3 & \in & \{0,1\}
\end{optprog*}

\begin{enumerate}
\item An inequality is called \textbf{valid} if adding it does not violate any of the constraints of the model. In other words, adding a valid inequality does not remove any current integer feasible points. Is the inequality $x_2 + x_3 \leq 2$ a valid inequality? Why or why not?
{\blu
Yes it is valid because it does not restrict the values $x_2$ and $x_3$ can take.
}
\item Is the inequality $x_1 + x_2 + x_3 \leq 3$ a valid inequality? Why or why not?
{
No it is not a valid inequality because it restricts the values that $x_1$ can take.
}
\item Suppose you solve the LP relaxation and obtain the solution $(x_1, x_2, x_3) = (4.75, 1, 0.25)$
	\begin{enumerate}
	\item What is the objective function value associated with this solution?
	{\blu
	11.5
	}
	\item Is this solution optimal for your IP?
	{\blu
	No, it's not integer
	}
	\end{enumerate}
\item Consider the solution $(4,1,1)$.
	\begin{enumerate}
	\item Is this solution feasible?
	{\blu
	Yes it does not violate any constraints.
	}
	\item What is the objective function value of this solution?
	{\blu
	7
	}
	\end{enumerate}
\item Using all of the information you've obtained so far, what are lower and upper bounds for the optimal objective function value of the IP, $z_{IP}$?
{\blu
\[
7 \leq z_{IP} \leq 11
\]
}
\item Suppose you replace the constraint $x_2 + x_3 \geq 1.25$ with $x_2 + x_3 \geq 2$
	\begin{enumerate}
	\item Explain why this is an appropriate constraint substitution.
	{
	\blu
	Because they are both binary, $\geq 1.25$ is the same as $\geq 2$
	}
	\item Suppose after solving the LP relaxation, you now obtain the solution $(4,1,1)$. Is this point optimal for the original IP? (Yes, No, Don't Know Yet). Why or why not?
	{
	\blu Yes it is because the new inequality is valid so our new solution is the real upper bound and the upper bound matches the lower bound so it must be optimal.
	}
	\end{enumerate}
\end{enumerate}

\item Is the set $S = \{(x,y): x^2 + y^2 \leq 4\}$ a convex set? Why or why not?

{\blu
Yes it is. If you graph it, the graph is a filled in circle.
}

\item Is the set $S = \{(x,y): 2x + y \leq 3, x \leq 2, x\geq 0 \text{ integer}, y \geq 0 \text{ integer}\}$ a convex set? Why or why not?
{\blu
No it's not. This is the feasible region of an IP and IPs are not convex sets.
}

\end{enumerate}


\end{document}
