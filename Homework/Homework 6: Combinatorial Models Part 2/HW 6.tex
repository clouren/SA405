\documentclass[11pt]{article}

%% MinionPro fonts 
%\usepackage[lf]{MinionPro}
%\usepackage{MnSymbol}
\usepackage{microtype}

%% Margins
\usepackage{geometry}
\geometry{verbose,letterpaper,tmargin=1in,bmargin=1in,lmargin=1in,rmargin=1in}

%% Other packages
\usepackage{amsmath}
\usepackage{amsthm}
\usepackage[shortlabels]{enumitem}
\usepackage{titlesec}
\usepackage{soul}
\usepackage{tikz}
\usepackage{mathtools}
\usepackage{pgfplots}
\usepackage{tikz-3dplot}
\usepackage{algorithmic}
\usepackage[export]{adjustbox}
\usepackage{tcolorbox}
\usepackage{optprog}

%% Paragraph style settings
\setlength{\parskip}{\medskipamount}
\setlength{\parindent}{0pt}

%% Change itemize bullets
\renewcommand{\labelitemi}{$\bullet$}
\renewcommand{\labelitemii}{$\circ$}
\renewcommand{\labelitemiii}{$\diamond$}
\renewcommand{\labelitemiv}{$\cdot$}

%% Colors
\definecolor{rred}{RGB}{204,0,0}
\definecolor{ggreen}{RGB}{0,145,0}
\definecolor{yyellow}{RGB}{255,185,0}
\definecolor{bblue}{rgb}{0.2,0.2,0.7}
\definecolor{ggray}{RGB}{190,190,190}
\definecolor{ppurple}{RGB}{160,32,240}
\definecolor{oorange}{RGB}{255,165,0}

%% Shrink section fonts
\titleformat*{\section}{\normalsize\bf}
\titleformat*{\subsection}{\normalsize\bf}
\titleformat*{\subsubsection}{\normalsize\it}

% %% Compress the spacing around section titles
\titlespacing*{\section}{0pt}{1.5ex}{0.75ex}
\titlespacing*{\subsection}{0pt}{1ex}{0.5ex}
\titlespacing*{\subsubsection}{0pt}{1ex}{0.5ex}

%% amsthm settings
\theoremstyle{definition}
\newtheorem{problem}{Problem}
\newtheorem{example}{Example}
\newtheorem*{theorem}{Theorem}
\newtheorem*{bigthm}{Big Theorem}
\newtheorem*{biggerthm}{Bigger Theorem}
\newtheorem*{bigcor1}{Big Corollary 1}
\newtheorem*{bigcor2}{Big Corollary 2}

%% tikz settings
\usetikzlibrary{calc}
\usetikzlibrary{patterns}
\usetikzlibrary{decorations}
\usepgfplotslibrary{polar}

%% algorithmic setup
\algsetup{linenodelimiter=}
\renewcommand{\algorithmiccomment}[1]{\quad// #1}
\renewcommand{\algorithmicrequire}{\emph{Input:}}
\renewcommand{\algorithmicensure}{\emph{Output:}}

%% Answer box macros
%% \answerbox{alignment}{width}{height}
\newcommand{\answerbox}[3]{%
  \fbox{%
    \begin{minipage}[#1]{#2}
      \hfill\vspace{#3}
    \end{minipage}
  }
}

%% \answerboxfull{alignment}{height}
\newcommand{\answerboxfull}[2]{%
  \answerbox{#1}{6.38in}{#2} 
}

%% \answerboxone{alignment}{height} -- for first-level bullet
\newcommand{\answerboxone}[2]{%
  \answerbox{#1}{6.0in}{#2} 
}

%% \answerboxtwo{alignment}{height} -- for second-level bullet
\newcommand{\answerboxtwo}[2]{%
  \answerbox{#1}{5.8in}{#2}
}

%% special boxes
\newcommand{\wordbox}{\answerbox{c}{1.2in}{.7cm}}
\newcommand{\catbox}{\answerbox{c}{.5in}{.7cm}}
\newcommand{\letterbox}{\answerbox{c}{.7cm}{.7cm}}

%% Miscellaneous macros
\newcommand{\tstack}[1]{\begin{multlined}[t] #1 \end{multlined}}
\newcommand{\cstack}[1]{\begin{multlined}[c] #1 \end{multlined}}
\newcommand{\ccite}[1]{\only<presentation>{{\scriptsize\color{gray} #1}}\only<article>{{\small [#1]}}}
\newcommand{\grad}{\nabla}
\newcommand{\ra}{\ensuremath{\rightarrow}~}
\newcommand{\maximize}{\text{maximize}}
\newcommand{\minimize}{\text{minimize}}
\newcommand{\subjectto}{\text{subject to}}
\newcommand{\trans}{\mathsf{T}}
\newcommand{\bb}{\mathbf{b}}
\newcommand{\bx}{\mathbf{x}}
\newcommand{\bc}{\mathbf{c}}
\newcommand{\bd}{\mathbf{d}}

%% LP format
%    \begin{align*}
%      \maximize \quad & \mathbf{c}^{\trans} \mathbf{x}\\
%      \subjectto \quad & A \mathbf{x} = \mathbf{b}\\
%                       & \mathbf{x} \ge \mathbf{0}
%    \end{align*}


%% Redefine maketitle
\makeatletter
\renewcommand{\maketitle}{
  \noindent SA405 -- AMP \hfill Rader \#2.42 \\

  \begin{center}\Large{\textbf{\@title}}\end{center}
}
\makeatother

%% ----- Begin document ----- %%
\begin{document}
  
\title{HW6: Combinatorial Models Part 2}


\maketitle

% 

\textbf{Problem 1:} ITSD is attempting to backup data on some external harddrives and have asked you for help. The data files have sizes 240, 462, 117, 560, 379, 110, 341, 294, 503, 469, 90, 65, 617, 500, 550, and 400 GB.
\begin{enumerate}
\item[a)] If the capacity of a harddrive is 780 GB, write a concrete integer programming model to determine the minimum number of harddrives needed to backup all files. \emph{Hint: You need two types of variables: one of which tells you which file is assigned to a hard drive and one of which tells you if a hard drive is used. Make sure you include logical constraints to relate these variables}.
\item[b)] Convert your concrete model to a parameterized model.
\item[c)] How many total variables does your model have?
\end{enumerate}

\textbf{Problem 2:} Consider the following graph:

\begin{center}
\includegraphics[width=0.5\textwidth]{Graph.png}
\end{center}

You are interested in coloring the nodes of this graph so that no node is adjacent to another node of the same color (thus for example, if node a is red, none of nodes b, d, or f can be red). Suppose that each node can be colored either red, teal, green, or yellow. 
\begin{enumerate}
\item [a)] Formulate an integer program whose solution would minimize the number of colors used. You can formulate as either a concrete or parameterized model. If you formulate a concrete model, do not try writing out all of the constraints, just write enough so that the pattern is clear. \emph{Hint: Just like in problem 1, you need two variables for this problem.}
\item [b)] How many total constraints does this model have?
\end{enumerate}







\end{document}
