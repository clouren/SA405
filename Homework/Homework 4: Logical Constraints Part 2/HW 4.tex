\documentclass[11pt]{article}

%% MinionPro fonts 
%\usepackage[lf]{MinionPro}
%\usepackage{MnSymbol}
\usepackage{microtype}

%% Margins
\usepackage{geometry}
\geometry{verbose,letterpaper,tmargin=1in,bmargin=1in,lmargin=1in,rmargin=1in}

%% Other packages
\usepackage{amsmath}
\usepackage{amsthm}
\usepackage[shortlabels]{enumitem}
\usepackage{titlesec}
\usepackage{soul}
\usepackage{tikz}
\usepackage{mathtools}
\usepackage{pgfplots}
\usepackage{tikz-3dplot}
\usepackage{algorithmic}
\usepackage[export]{adjustbox}
\usepackage{tcolorbox}
\usepackage{optprog}

%% Paragraph style settings
\setlength{\parskip}{\medskipamount}
\setlength{\parindent}{0pt}

%% Change itemize bullets
\renewcommand{\labelitemi}{$\bullet$}
\renewcommand{\labelitemii}{$\circ$}
\renewcommand{\labelitemiii}{$\diamond$}
\renewcommand{\labelitemiv}{$\cdot$}

%% Colors
\definecolor{rred}{RGB}{204,0,0}
\definecolor{ggreen}{RGB}{0,145,0}
\definecolor{yyellow}{RGB}{255,185,0}
\definecolor{bblue}{rgb}{0.2,0.2,0.7}
\definecolor{ggray}{RGB}{190,190,190}
\definecolor{ppurple}{RGB}{160,32,240}
\definecolor{oorange}{RGB}{255,165,0}

%% Shrink section fonts
\titleformat*{\section}{\normalsize\bf}
\titleformat*{\subsection}{\normalsize\bf}
\titleformat*{\subsubsection}{\normalsize\it}

% %% Compress the spacing around section titles
\titlespacing*{\section}{0pt}{1.5ex}{0.75ex}
\titlespacing*{\subsection}{0pt}{1ex}{0.5ex}
\titlespacing*{\subsubsection}{0pt}{1ex}{0.5ex}

%% amsthm settings
\theoremstyle{definition}
\newtheorem{problem}{Problem}
\newtheorem{example}{Example}
\newtheorem*{theorem}{Theorem}
\newtheorem*{bigthm}{Big Theorem}
\newtheorem*{biggerthm}{Bigger Theorem}
\newtheorem*{bigcor1}{Big Corollary 1}
\newtheorem*{bigcor2}{Big Corollary 2}

%% tikz settings
\usetikzlibrary{calc}
\usetikzlibrary{patterns}
\usetikzlibrary{decorations}
\usepgfplotslibrary{polar}

%% algorithmic setup
\algsetup{linenodelimiter=}
\renewcommand{\algorithmiccomment}[1]{\quad// #1}
\renewcommand{\algorithmicrequire}{\emph{Input:}}
\renewcommand{\algorithmicensure}{\emph{Output:}}

%% Answer box macros
%% \answerbox{alignment}{width}{height}
\newcommand{\answerbox}[3]{%
  \fbox{%
    \begin{minipage}[#1]{#2}
      \hfill\vspace{#3}
    \end{minipage}
  }
}

%% \answerboxfull{alignment}{height}
\newcommand{\answerboxfull}[2]{%
  \answerbox{#1}{6.38in}{#2} 
}

%% \answerboxone{alignment}{height} -- for first-level bullet
\newcommand{\answerboxone}[2]{%
  \answerbox{#1}{6.0in}{#2} 
}

%% \answerboxtwo{alignment}{height} -- for second-level bullet
\newcommand{\answerboxtwo}[2]{%
  \answerbox{#1}{5.8in}{#2}
}

%% special boxes
\newcommand{\wordbox}{\answerbox{c}{1.2in}{.7cm}}
\newcommand{\catbox}{\answerbox{c}{.5in}{.7cm}}
\newcommand{\letterbox}{\answerbox{c}{.7cm}{.7cm}}

%% Miscellaneous macros
\newcommand{\tstack}[1]{\begin{multlined}[t] #1 \end{multlined}}
\newcommand{\cstack}[1]{\begin{multlined}[c] #1 \end{multlined}}
\newcommand{\ccite}[1]{\only<presentation>{{\scriptsize\color{gray} #1}}\only<article>{{\small [#1]}}}
\newcommand{\grad}{\nabla}
\newcommand{\ra}{\ensuremath{\rightarrow}~}
\newcommand{\maximize}{\text{maximize}}
\newcommand{\minimize}{\text{minimize}}
\newcommand{\subjectto}{\text{subject to}}
\newcommand{\trans}{\mathsf{T}}
\newcommand{\bb}{\mathbf{b}}
\newcommand{\bx}{\mathbf{x}}
\newcommand{\bc}{\mathbf{c}}
\newcommand{\bd}{\mathbf{d}}

%% LP format
%    \begin{align*}
%      \maximize \quad & \mathbf{c}^{\trans} \mathbf{x}\\
%      \subjectto \quad & A \mathbf{x} = \mathbf{b}\\
%                       & \mathbf{x} \ge \mathbf{0}
%    \end{align*}


%% Redefine maketitle
\makeatletter
\renewcommand{\maketitle}{
  \noindent SA405 -- AMP \hfill Rader \#2.42 \\

  \begin{center}\Large{\textbf{\@title}}\end{center}
}
\makeatother

%% ----- Begin document ----- %%
\begin{document}
  
\title{HW4: Logical Constraints Part 2}


\maketitle

% Either or

\textbf{Problem 1:} Mazda is considering expanding their production capabilities in a factory. The factory currently makes three types of cars: Mazda3 (the best one obviously), Mazda6, and Mazda CX5. Each car has a requirement of steel, labor, and expected profit given in the table below:

\begin{center}
\begin{tabular}{c|ccc} \hline
Resource & Mazda3   & Mazda6   & Mazda CX5 \\ \hline
Steel    & 1.5 tons & 2 tons   & 3 tons  \\
Labor    & 30 hours & 28 hours & 40 hours \\
Profit   & \$2000   & \$2500   & \$3500 \\ \hline
\end{tabular}
\end{center}

Mazda currently has 6000 tons of steel and 60,000 hours of labor at the plant. Lastly, Mazda does not consider it economically viable to produce Mazda CX5 unless at least 700 of them are produced.

\begin{enumerate}
\item[a.] Formulate an integer program that would allow Mazda to maximize their profit.
\item[b.] Mazda has the option to hire part time help. In total, they could obtain 5000 new hours of labor for the price of \$50,000. Modify your model from part (a) to determine if Mazda should make this change.
\item[c.] Mazda is considering using lightweight materials to reduce the amount of steel used. To reconfigure their machines, it would cost \$25,000; but would reduce the amount of steel needed for a Mazda3, Mazda6, and Mazda CX5 to be 1.25 tons, 1.8 tons, and 2.85 tons, respectively. Again, modify your model from part (a) to determine if they should make this change.
\end{enumerate}

\newpage

% If then

\textbf{Problem 2:} Professor Lourenco is planning a trip to Europe. There are eight locations he's considering to visit. Each location has a cost to visit and level of happiness (from 1 to 10) it would bring him given in the table below.

\begin{center}
\begin{tabular}{c|cc} \hline
Location   & Cost   & Happiness \\ \hline
Barcelona  & \$1200 & 8    \\
Lisbon     & \$1000 & 8.5  \\
The Azores & \$800  & 9   \\
London     & \$1300 & 7   \\
Rome       & \$1500 & 10   \\
Paris      & \$1000 & 6    \\
Florence   & \$950  & 7.5  \\
Oslo       & \$1100 & 8    \\ \hline
\end{tabular}
\end{center}

\begin{enumerate}
\item[a.] If Professor Lourenco has a budget of \$5500, formulate an integer program to help him decide where to go in order to maximize his happiness.
\item[b.] Professor Lourenco has decided to add some extra stipulations to his trip. Specifically:
	\begin{itemize}
	\item He doesn't want to see only huge cities, so he can only go to at most three of Barcelona, Lisbon, London, Rome, and Paris.
	\item If he visits both the Azores and Lisbon, he doesn't want to visit Barcelona.
	\item If he does not visit Rome, he wants to visit Florence.
	\item If he visits the Azores he wants to also visit Rome and Florence.
	\end{itemize}
Using the variables you defined in part a, write constraints to enforce each of these requirements.
\end{enumerate}

\newpage

\textbf{Problem 3: OPTIONAL/BONUS} You don't have to turn this one in, but if you get it correct you'll get some bonus points. Consider the following integer program:

\begin{optprog*}
max & \objective{x_1+x_2} \\
st & x_1 + x_2 & \leq & 8 \\
   & -x_1 + x_2 & \leq & 2 \\
   & x_1 & \in & \{0,1,4,6\} \\
   & x_2 & \geq & 0
\end{optprog*}

Reformulate this integer program to \textbf{only} use $\geq 0$ and binary variables.



\end{document}
