\documentclass[11pt]{article}

%% MinionPro fonts 
%\usepackage[lf]{MinionPro}
%\usepackage{MnSymbol}
\usepackage{microtype}

%% Margins
\usepackage{geometry}
\geometry{verbose,letterpaper,tmargin=1in,bmargin=1in,lmargin=1in,rmargin=1in}

%% Other packages
\usepackage{amsmath}
\usepackage{amsthm}
\usepackage[shortlabels]{enumitem}
\usepackage{titlesec}
\usepackage{soul}
\usepackage{tikz}
\usepackage{mathtools}
\usepackage{pgfplots}
\usepackage{tikz-3dplot}
\usepackage{algorithmic}
\usepackage[export]{adjustbox}
\usepackage{tcolorbox}
\usepackage{optprog}
\usepackage{amsfonts}
\usepackage{pdfpages}

%% Paragraph style settings
\setlength{\parskip}{\medskipamount}
\setlength{\parindent}{0pt}

%% Change itemize bullets
\renewcommand{\labelitemi}{$\bullet$}
\renewcommand{\labelitemii}{$\circ$}
\renewcommand{\labelitemiii}{$\diamond$}
\renewcommand{\labelitemiv}{$\cdot$}

%% Colors
\definecolor{rred}{RGB}{204,0,0}
\definecolor{ggreen}{RGB}{0,145,0}
\definecolor{yyellow}{RGB}{255,185,0}
\definecolor{bblue}{rgb}{0.2,0.2,0.7}
\definecolor{ggray}{RGB}{190,190,190}
\definecolor{ppurple}{RGB}{160,32,240}
\definecolor{oorange}{RGB}{255,165,0}

%% Shrink section fonts
\titleformat*{\section}{\normalsize\bf}
\titleformat*{\subsection}{\normalsize\bf}
\titleformat*{\subsubsection}{\normalsize\it}

% %% Compress the spacing around section titles
\titlespacing*{\section}{0pt}{1.5ex}{0.75ex}
\titlespacing*{\subsection}{0pt}{1ex}{0.5ex}
\titlespacing*{\subsubsection}{0pt}{1ex}{0.5ex}

%% amsthm settings
\theoremstyle{definition}
\newtheorem{problem}{Problem}
\newtheorem{example}{Example}
\newtheorem*{theorem}{Theorem}
\newtheorem*{bigthm}{Big Theorem}
\newtheorem*{biggerthm}{Bigger Theorem}
\newtheorem*{bigcor1}{Big Corollary 1}
\newtheorem*{bigcor2}{Big Corollary 2}

%% tikz settings
\usetikzlibrary{calc}
\usetikzlibrary{patterns}
\usetikzlibrary{decorations}
\usepgfplotslibrary{polar}


%% algorithmic setup
\algsetup{linenodelimiter=}
\renewcommand{\algorithmiccomment}[1]{\quad// #1}
\renewcommand{\algorithmicrequire}{\emph{Input:}}
\renewcommand{\algorithmicensure}{\emph{Output:}}

%% Answer box macros
%% \answerbox{alignment}{width}{height}
\newcommand{\answerbox}[3]{%
  \fbox{%
    \begin{minipage}[#1]{#2}
      \hfill\vspace{#3}
    \end{minipage}
  }
}

%% \answerboxfull{alignment}{height}
\newcommand{\answerboxfull}[2]{%
  \answerbox{#1}{6.38in}{#2} 
}

%% \answerboxone{alignment}{height} -- for first-level bullet
\newcommand{\answerboxone}[2]{%
  \answerbox{#1}{6.0in}{#2} 
}

%% \answerboxtwo{alignment}{height} -- for second-level bullet
\newcommand{\answerboxtwo}[2]{%
  \answerbox{#1}{5.8in}{#2}
}

%% special boxes
\newcommand{\wordbox}{\answerbox{c}{1.2in}{.7cm}}
\newcommand{\catbox}{\answerbox{c}{.5in}{.7cm}}
\newcommand{\letterbox}{\answerbox{c}{.7cm}{.7cm}}

%% Miscellaneous macros
\newcommand{\tstack}[1]{\begin{multlined}[t] #1 \end{multlined}}
\newcommand{\cstack}[1]{\begin{multlined}[c] #1 \end{multlined}}
\newcommand{\ccite}[1]{\only<presentation>{{\scriptsize\color{gray} #1}}\only<article>{{\small [#1]}}}
\newcommand{\grad}{\nabla}
\newcommand{\ra}{\ensuremath{\rightarrow}~}
\newcommand{\maximize}{\text{maximize}}
\newcommand{\minimize}{\text{minimize}}
\newcommand{\subjectto}{\text{subject to}}
\newcommand{\trans}{\mathsf{T}}
\newcommand{\bb}{\mathbf{b}}
\newcommand{\bx}{\mathbf{x}}
\newcommand{\bc}{\mathbf{c}}
\newcommand{\bd}{\mathbf{d}}
\newcommand{\blu}{\color{blue}}
%% LP format
%    \begin{align*}
%      \maximize \quad & \mathbf{c}^{\trans} \mathbf{x}\\
%      \subjectto \quad & A \mathbf{x} = \mathbf{b}\\
%                       & \mathbf{x} \ge \mathbf{0}
%    \end{align*}


%% Redefine maketitle
\makeatletter
\renewcommand{\maketitle}{
  \noindent SA405 -- AMP 

  \begin{center}\Large{\textbf{\@title}}\end{center}
}
\makeatother

%% ----- Begin document ----- %%
\begin{document}

\begin{large}
\begin{center}
\textbf{Lesson 10/11 HW: Combinatorial Optimization Models}
\end{center}
\end{large}

ITSD is attempting to backup data on some external harddrives and have asked you for help. The data files have sizes 240, 462, 117, 560, 379, 110, 341, 294, 503, 469, 90, 65, 617, 500, 550, and 400 GB.
\begin{enumerate}
\item[a)] If the capacity of a harddrive is 780 GB, write a concrete integer programming model to determine the minimum number of harddrives needed to backup all files. \emph{Hint: You need two types of variables. One is a $x_{f,h}$ variable to which equals 1 if a file is assigned to a harddrive. The other is a $z_h$ which equals 1 if that harddrive is used. Make sure you include logical constraints to relate these variables}.
\item[b)] Convert your concrete model to a parameterized model.
\item[c)] How many total variables does your model have?
\item[d)] \textbf{Optional:} Implement your model in Python. You may have to start with a smaller number of harddrives and gradually increase. This is an interesting example of how for even a ``small'' problem, a solver can be slowed down with some difficult constraints/variables.
\end{enumerate}

  
{\blu Solution }

Part (a)

{
\blu

In this problem, we don't know the total number of Harddrives that we will need ahead of time. We can make a rough guess: the sum of all sizes is:
\[
240+462+\cdots + 400 = 5697 
\]
\[
5697 / 780 = 7.3
\]

This tells us that we will need at least 8 Harddrives (why?). I'll model it with 12 CDs to be safe. I'll do an abbreviated concrete model.

\textbf{\underline{Sets}} \\

Let $F$ be the set of files (specifically $F = \{1, 2, 3, 4 \cdots 16\}$ \\
Let $H$ be the set of Harddrives (specifically $H = \{1, 2, 3 , \cdots, 12 \}$ \\


\textbf{\underline{Decision Variables}}

Let $x_{f,h} = 1$ if file $f \in F$ is placed on harddrive $h \in H$ or $0$ otherwise. \\
Let $z_h = 1$ if harddrive $h \in H$ is used.

\textbf{\underline{Parameters}} \\
none \\

\textbf{\underline{Objective Function}}

Our goal is to minimize the total number of harddrives used, so I'll minimize the sum of $z_h$

\[
\text{minimize: } z_1 + z_2 + \cdots z_{12}
\]

\textbf{\underline{Constraints}}

Logically, we need three types of constraints:
	\begin{enumerate}
	\item Every file is placed on a harddrive
	\item No harddrive is overloaded
	\item If a file is placed on a harddrive, that harddrive is used.
	\end{enumerate}

To be super clear in the concrete model, I'll formulate it with these three types of constraints. Note that it is possible to combine constraints 2 and 3 and only have two types of constraints in your final model. That's what I'll do for the parameterized model

\begin{optprog*}
& x_{1,1} + x_{1,2} + \ldots x_{1,12} & = & 1 & & \text{(File 1 goes on a harddrive)} \\
& x_{2,1} + x_{2,2} + \ldots x_{2,12} & = & 1 & & \text{(File 2 goes on a harddrive)} \\
& \vdots \\
& x_{16,1} + x_{16,2} + \ldots x_{16,12} & = & 1 & & \text{(File 16 goes on a harddrive)} \\
& 240 x_{1,1} + 462 x_{2,1} + \ldots 396 x_{16,1} &  \leq & 780 & & \text{(harddrive 1 is not overloaded)} \\
& 240 x_{1,2} + 462 x_{2,2} + \ldots 396 x_{16,2} &  \leq & 780 & & \text{(harddrive 2 is not overloaded)} \\
& \vdots \\
& 240 x_{1,12} + 462 x_{2,12} + \ldots 396 x_{16,12} &  \leq & 780 & & \text{(harddrive 12 is not overloaded)} \\
& x_{1,1} + x_{2,1} + \ldots + x_{16,1} & \leq & 16 z_1 & & \text{(harddrive 1 must be used if a file is added to it)} \\
& x_{1,2} + x_{2,2} + \ldots + x_{16,2} & \leq & 16 z_2 & & \text{(harddrive 2 must be used if a file is added to it)} \\
& \vdots \\
& x_{1,12} + x_{2,12} + \ldots + x_{16,12} & \leq & 16 z_{12} & & \text{(harddrive 12 must be used if a file is added to it)} \\
& x_{f,h} , z_h & \in & \{0,1\} & \text{for all $f \in F, h \in H$} & \text{(binary)}
\end{optprog*}

}


\textbf{Part (b)}

{
\blu


\textbf{\underline{Sets}} \\

Let $F$ be the set of files \\
Let $H$ be the set of harddrives  \\


\textbf{\underline{Decision Variables}}

Let $x_{f,h} = 1$ if file $f \in F$ is placed on harddrive $h \in H$ or $0$ otherwise. \\
Let $z_h = 1$ if harddrive $h \in H$ is used. \\

\textbf{\underline{Parameters}} \\
Let $s_f$ be the size of file $f$ \\
Let $M$ be the maximum capacity of a harddrive \\


\textbf{\underline{Objective Function}}

\[
\text{minimize: } \sum_{h \in H} z_h
\]

\textbf{\underline{Constraints}}

Here in the parameterized model, I will combine the overloaded and linking constraints so my final model will only have 2 types of constraints (not including binary)

\begin{optprog*}
& \sum_{h \in H} x_{f,h} & = & 1 & \text{for all $f \in F$} & \text{(file $f$ goes on a harddrive)} \\
& \sum_{f \in F} s_f x_{f,h} & \leq & M z_h & \text{for all $h \in H$} & \text{(Capacity and linking)} \\
& x_{f,h} , z_h & \in & \{0,1\} & \text{for all $f \in F, h \in H$} & \text{(binary)}
\end{optprog*}

}

\textbf{Part c}

{
\blu
I have one $x$ variable for every file/harddrive combination. Since I used 12 harddrives, this leaves me with 
\[
16*12 = 192 \text{ total $x$ variables}
\]

Additionally, I have 1 $z$ variable for every harddrive, thus I have a total of 12 $z$ variables. Thus, I have a grand total of $204$ variables.

Bonus: How many constraints? In my abstract model, I only have $16+12=28$ total constraints (excluding binary constraints). However, this is a hard problem to solve due to the massive number of variables (imagine how much a problem like this can blow up for a large problem).

}

\textbf{Part d} I did not implement this in python. 


\end{document}
