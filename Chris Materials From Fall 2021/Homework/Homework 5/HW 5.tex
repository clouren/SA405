\documentclass[11pt]{article}

%% MinionPro fonts 
%\usepackage[lf]{MinionPro}
%\usepackage{MnSymbol}
\usepackage{microtype}

%% Margins
\usepackage{geometry}
\geometry{verbose,letterpaper,tmargin=1in,bmargin=1in,lmargin=1in,rmargin=1in}

%% Other packages
\usepackage{amsmath}
\usepackage{amsthm}
\usepackage[shortlabels]{enumitem}
\usepackage{titlesec}
\usepackage{soul}
\usepackage{tikz}
\usepackage{mathtools}
\usepackage{pgfplots}
\usepackage{tikz-3dplot}
\usepackage{algorithmic}
\usepackage[export]{adjustbox}
\usepackage{tcolorbox}
\usepackage{optprog}
\usepackage{amsfonts}
\usepackage{pdfpages}

%% Paragraph style settings
\setlength{\parskip}{\medskipamount}
\setlength{\parindent}{0pt}

%% Change itemize bullets
\renewcommand{\labelitemi}{$\bullet$}
\renewcommand{\labelitemii}{$\circ$}
\renewcommand{\labelitemiii}{$\diamond$}
\renewcommand{\labelitemiv}{$\cdot$}

%% Colors
\definecolor{rred}{RGB}{204,0,0}
\definecolor{ggreen}{RGB}{0,145,0}
\definecolor{yyellow}{RGB}{255,185,0}
\definecolor{bblue}{rgb}{0.2,0.2,0.7}
\definecolor{ggray}{RGB}{190,190,190}
\definecolor{ppurple}{RGB}{160,32,240}
\definecolor{oorange}{RGB}{255,165,0}

%% Shrink section fonts
\titleformat*{\section}{\normalsize\bf}
\titleformat*{\subsection}{\normalsize\bf}
\titleformat*{\subsubsection}{\normalsize\it}

% %% Compress the spacing around section titles
\titlespacing*{\section}{0pt}{1.5ex}{0.75ex}
\titlespacing*{\subsection}{0pt}{1ex}{0.5ex}
\titlespacing*{\subsubsection}{0pt}{1ex}{0.5ex}

%% amsthm settings
\theoremstyle{definition}
\newtheorem{problem}{Problem}
\newtheorem{example}{Example}
\newtheorem*{theorem}{Theorem}
\newtheorem*{bigthm}{Big Theorem}
\newtheorem*{biggerthm}{Bigger Theorem}
\newtheorem*{bigcor1}{Big Corollary 1}
\newtheorem*{bigcor2}{Big Corollary 2}

%% tikz settings
\usetikzlibrary{calc}
\usetikzlibrary{patterns}
\usetikzlibrary{decorations}
\usepgfplotslibrary{polar}


%% algorithmic setup
\algsetup{linenodelimiter=}
\renewcommand{\algorithmiccomment}[1]{\quad// #1}
\renewcommand{\algorithmicrequire}{\emph{Input:}}
\renewcommand{\algorithmicensure}{\emph{Output:}}

%% Answer box macros
%% \answerbox{alignment}{width}{height}
\newcommand{\answerbox}[3]{%
  \fbox{%
    \begin{minipage}[#1]{#2}
      \hfill\vspace{#3}
    \end{minipage}
  }
}

%% \answerboxfull{alignment}{height}
\newcommand{\answerboxfull}[2]{%
  \answerbox{#1}{6.38in}{#2} 
}

%% \answerboxone{alignment}{height} -- for first-level bullet
\newcommand{\answerboxone}[2]{%
  \answerbox{#1}{6.0in}{#2} 
}

%% \answerboxtwo{alignment}{height} -- for second-level bullet
\newcommand{\answerboxtwo}[2]{%
  \answerbox{#1}{5.8in}{#2}
}

%% special boxes
\newcommand{\wordbox}{\answerbox{c}{1.2in}{.7cm}}
\newcommand{\catbox}{\answerbox{c}{.5in}{.7cm}}
\newcommand{\letterbox}{\answerbox{c}{.7cm}{.7cm}}

%% Miscellaneous macros
\newcommand{\tstack}[1]{\begin{multlined}[t] #1 \end{multlined}}
\newcommand{\cstack}[1]{\begin{multlined}[c] #1 \end{multlined}}
\newcommand{\ccite}[1]{\only<presentation>{{\scriptsize\color{gray} #1}}\only<article>{{\small [#1]}}}
\newcommand{\grad}{\nabla}
\newcommand{\ra}{\ensuremath{\rightarrow}~}
\newcommand{\maximize}{\text{maximize}}
\newcommand{\minimize}{\text{minimize}}
\newcommand{\subjectto}{\text{subject to}}
\newcommand{\trans}{\mathsf{T}}
\newcommand{\bb}{\mathbf{b}}
\newcommand{\bx}{\mathbf{x}}
\newcommand{\bc}{\mathbf{c}}
\newcommand{\bd}{\mathbf{d}}
\newcommand{\blu}{\color{blue}}
%% LP format
%    \begin{align*}
%      \maximize \quad & \mathbf{c}^{\trans} \mathbf{x}\\
%      \subjectto \quad & A \mathbf{x} = \mathbf{b}\\
%                       & \mathbf{x} \ge \mathbf{0}
%    \end{align*}


%% Redefine maketitle
\makeatletter
\renewcommand{\maketitle}{
  \noindent SA405 -- AMP 

  \begin{center}\Large{\textbf{\@title}}\end{center}
}
\makeatother

%% ----- Begin document ----- %%
\begin{document}

\begin{large}
\begin{center}
\textbf{HW 5: Unsymmetric Traveling Salesperson problem}
\end{center}
\end{large}

The TSP problem we did in class operated on an undirected graph; thus, an edge like $(1,2)$ can be from node 1 to node 2 or node 2 to node 1. In this homework problem, we will consider the unsymmetric traveling salesperson problem which operates on a directed graph (thus edge $(1,2)$ and $(2,1)$ are distinct).

\begin{problem}
Cameron is taking a vacation to New York City and wants to drive around to see some of the famous landmarks there. There are 8 total landmarks he wants to see. Since many of the roads in NYC are one way only, the path between two landmarks is not the same distance. The matrix below gives the distances (in miles) between each of the 8 landmarks he wants to visit.

\[
D = 
\begin{bmatrix}
0 & 2 & 3 & 1 & 7 & 5 & 4 & 4\\
1 & 0 & 3 & 2 & 4 & 2 & 6 & 3\\
2 & 1 & 0 & 4 & 2 & 4 & 7 & 1\\
3 & 2 & 1 & 0 & 4 & 3 & 2 & 2\\
6 & 8 & 1 & 2 & 0 & 4 & 6 & 5\\
3 & 2 & 4 & 2 & 7 & 0 & 3 & 4\\
2 & 3 & 1 & 4 & 5 & 6 & 0 & 2\\
1 & 4 & 3 & 2 & 7 & 4 & 3 & 0
\end{bmatrix}
\]

He wants to visit all of the landmarks while driving as little as possible and he asks you for help.
\end{problem}

\begin{enumerate}
\item Formulate a concrete model whose solution will give Cameron a tour that visits all 8 landmarks exactly once. \emph{Hint: Do not write all the subtour elimination constraints, you can write one or two then move on. Also, if you're having trouble getting started draw out the network for the problem.}
\item Convert your concrete model above to a parameterized model.
\item Suppose after solving your model, the solver returns the following solution. 

\verb|The following edges should be selected: (1,6), (2,1), (3,7), (4,3),| 

\verb|(5,2), (6,5), (7,8), (8,4)|
	\begin{enumerate}
	\item What are the values of your variables associated with this solution?
	\item What is the total distance traveled by this solution?
	\item Is this solution optimal for your TSP problem?
	\item If the solution is not optimal, write a constraint you could add to your model to remove this solution from your feasible region.
	\end{enumerate}
\end{enumerate}


\end{document}
