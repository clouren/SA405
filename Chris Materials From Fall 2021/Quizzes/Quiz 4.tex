\documentclass[letterpaper,oneside,12pt]{article}%twocolumn;titlepage(separate title page)
\usepackage{pslatex}                %Times New Roman Font; Improves on-screen readability; preferred to using times package
\usepackage[margin=0.5in]{geometry}   %1in 'uniform' margin (with oneside)
\usepackage{color}
\usepackage{amsmath}
\usepackage{amsfonts}
\usepackage{enumitem}
\usepackage{url}
\usepackage{bm}
\usepackage{graphicx}
\usepackage{optprog}
\usepackage{lscape}
\setcounter{MaxMatrixCols}{20}
%\setlength{\headsep}{0in}           %Distance from bottom of header to the body of text on a page.
\newcommand{\rem}[1]{}
\renewcommand{\arraystretch}{1}

% **********************************************************************
% Compact itemize, enumerate and description environments %pkg enumitem
% **********************************************************************
\usepackage{enumitem}            % [FAILS w/BEAMER] Lets define enumerate spacing: \begin{enumerate}[topsep=0pt,itemsep=0pt,parsep=0pt,partopsep=0pt]
\newenvironment{compitem}{\begin{itemize}[topsep=0pt,itemsep=0pt,parsep=0pt,partopsep=0pt]}{\end{itemize}}
\newenvironment{compenum}{\begin{enumerate}[topsep=0pt,itemsep=0pt,parsep=0pt,partopsep=0pt]}{\end{enumerate}}
\newenvironment{compdesc}{\begin{description}[topsep=0pt,itemsep=0pt,parsep=0pt,partopsep=0pt]}{\end{description}}

\DeclareMathOperator{\conv}{conv}
\DeclareMathOperator*{\proj}{proj}

\newcommand{\blu}{\color{blue}}
\newcommand{\bla}{\color{black}}
\begin{document}
\noindent\fbox{%
	\parbox{\textwidth}{%
		\begin{center}
		\textbf{\large{Department of Mathematics}} \\
		\textbf{\large{SA 405 - Advanced Mathematical Programming}} \\
		\textbf{\large{Quiz 4}}
		\end{center}
	}%
}

\vspace{3mm} \hspace{\fill} \textbf{Name: \underline{\hspace{6cm}}}

You're going to Six Flags this weekend and are a roller coaster enthusiast. Six Flags has 7 total roller coasters. You label them 1 to 7 and plot out the distance you'd have to walk in order to get to each roller-coaster. The following matrix gives these distances (note it's symmetric):
\[
\begin{bmatrix}
-& 10 & 12 & 7  & 13 & 8  & 9 \\
 & -  & 10 & 13 & 25 & 6  & 7 \\
 &    & -  & 21 & 23 & 12 & 10 \\
 &    &    & -  & 12 & 3  & 9   \\
 &    &    &    & -  & 20 & 8   \\
 &    &    &    &    & -  & 3   \\
 &    &    &    &    &    & -
\end{bmatrix}
\]

You decide that you want to walk as little as possible; so you model this problem as a minimum spanning tree (MST) in order to find the path you'd have to take to walk as little as possible.

Suppose you use the variables $x_{i,j} = 1$ if edge $(i,j)$ is part of the tree and 0 otherwise.

\begin{enumerate}
\item (20 points) When modeling this problem as a MST you need to include a constraint that says you visit each roller coaster. Using the $x_{i,j}$ variables, write a concrete constraint that ensures at least one edge connected to roller coaster 3 is selected. \vspace{1.5in}

\item After implementing this model in python, your solver returns the following solution:

	\verb|The optimal solution is to select edges [1,2], [2,3], [3,5], [4,6], [4,7], [6,7]|
		\begin{enumerate}
		\item (15 points) What are the values of the $x_{i,j}$ variables corresponding to this solution? \vfill
		\item (15 points) What is the total distanced traveled by this solution? \vfill \newpage
		\item (10 points) Is this solution optimal to your MST problem? \vspace{1in}
		\item (20 points) If this solution is optimal explain why. If the solution is not optimal, write which constraint(s) you would add to your model to exclude this solution. \vspace{2in}
		\end{enumerate}
\end{enumerate}

\begin{enumerate}[resume]
\item (10 points) Given a graph on $n$ nodes, how many edges do I need to choose to form a cycle? \vspace{1in}
\item (10 points) True or False, when solving a MST problem in python, we should implement \textbf{every} constraint in the model. \vspace{1in}
\end{enumerate}



\end{document}
