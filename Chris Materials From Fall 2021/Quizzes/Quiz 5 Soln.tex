\documentclass[letterpaper,oneside,12pt]{article}%twocolumn;titlepage(separate title page)
\usepackage{pslatex}                %Times New Roman Font; Improves on-screen readability; preferred to using times package
\usepackage[margin=0.5in]{geometry}   %1in 'uniform' margin (with oneside)
\usepackage{color}
\usepackage{amsmath}
\usepackage{amsfonts}
\usepackage{enumitem}
\usepackage{url}
\usepackage{bm}
\usepackage{graphicx}
\usepackage{optprog}
\usepackage{lscape}
\setcounter{MaxMatrixCols}{20}
%\setlength{\headsep}{0in}           %Distance from bottom of header to the body of text on a page.
\newcommand{\rem}[1]{}
\renewcommand{\arraystretch}{1}

% **********************************************************************
% Compact itemize, enumerate and description environments %pkg enumitem
% **********************************************************************
\usepackage{enumitem}            % [FAILS w/BEAMER] Lets define enumerate spacing: \begin{enumerate}[topsep=0pt,itemsep=0pt,parsep=0pt,partopsep=0pt]
\newenvironment{compitem}{\begin{itemize}[topsep=0pt,itemsep=0pt,parsep=0pt,partopsep=0pt]}{\end{itemize}}
\newenvironment{compenum}{\begin{enumerate}[topsep=0pt,itemsep=0pt,parsep=0pt,partopsep=0pt]}{\end{enumerate}}
\newenvironment{compdesc}{\begin{description}[topsep=0pt,itemsep=0pt,parsep=0pt,partopsep=0pt]}{\end{description}}

\DeclareMathOperator{\conv}{conv}
\DeclareMathOperator*{\proj}{proj}

\newcommand{\blu}{\color{blue}}
\newcommand{\bla}{\color{black}}
\begin{document}
\noindent\fbox{%
	\parbox{\textwidth}{%
		\begin{center}
		\textbf{\large{Department of Mathematics}} \\
		\textbf{\large{SA 405 - Advanced Mathematical Programming}} \\
		\textbf{\large{Quiz 5}}
		\end{center}
	}%
}

\vspace{3mm} \hspace{\fill} \textbf{Name: \underline{\hspace{6cm}}}

You are continuing your roller coaster enthusiasm by planning a trip to Busch Gardens. Busch Gardens has 8 roller coasters and, yet again, you label them 1 to 8 and plot the distance you'd have to walk in order to get to each roller coaster. The following matrix gives these distances (in hundreds of feet).

\[
\begin{bmatrix}
-& 6  & 8  & 3  & 12 & 5  & 7 & 6  \\
 & -  & 6  & 4  & 6  & 4  & 3 & 2  \\
 &    & -  & 13 & 10 & 2  & 7 & 10 \\
 &    &    & -  & 12 & 3  & 9 & 13 \\
 &    &    &    & -  & 6  & 9 & 12 \\
 &    &    &    &    & -  & 7 & 4  \\
 &    &    &    &    &    & - & 1  \\
 &    &    &    &    &    &   & -
\end{bmatrix}
\]

You've refined your model and decided to model this problem as a traveling salesperson (TSP) in order to walk as little as possible.

Suppose you use the variables $x_{i,j} = 1$ if edge $(i,j)$ is part of the tour and 0 otherwise.

\begin{enumerate}
\item (15 points) Give the objective function of this model in both concrete and parameterized form. For the parameterized form, make sure you define any new parameters used. 

{\blu
\[
\text{min } 6 x_{1,2}+8 x_{1,3} + \cdots + x_{7,8}
\]

For the parameterized form, define $d_{i,j}$ as the distance along edge $(i,j)$ for all $(i,j) \in E$. Then, the objective is:
\[
\text{min } \sum_{(i,j) \in E} d_{i,j} x_{i,j}
\]

}

\item (15 points) In order to obtain a tour of the graph, how many edges must be selected? Write a constraint, in either concrete or parameterized form, which enforces that this number of edges is selected from the graph. 

{\blu
You must select 8 edges to form a tour since there are 8 nodes. This constraint is:

\[
\sum_{(i,j) \in E} x_{i,j} = 8
\]
}


\newpage

\item You implement this model in python, solve it, and get the following solution:

	\verb|The optimal solution is to select cycles 2-1-5-3-2 and 4-7-8-6-4|
		\begin{enumerate}
		\item (15 points) What are the values of the $x_{i,j}$ variables corresponding to this solution? 

{\blu
\[
x_{1,2}=x_{1,5}=x_{3,5}=x_{2,3}=x_{4,7}=x_{7,8} = x_{6,8} = x_{4,6} = 1
\]

All other $x_{i,j}$ are 0.

}

		\vfill
		\item (15 points) What is the total distanced traveled by this solution?  

{\blu

\[
6+12+10+6+9+1+4+3=51
\]
}		\vfill
		
		
		\item You know that this is not the optimal solution to your problem because it is not a tour of the entire graph, but instead is two cycles of size 4. You decide to eliminate the first cycle \verb|2-1-5-3-2|. You recall that the general subtour elimination constraints for TSP were:
		\[
		\sum_{(i,j) \in E: \\ i \in S, j \in S} x_{i,j} \leq |S| - 1 \text{ for all $S \subset N$, $|S| \geq 3$}
		\]
			\begin{enumerate}
			\item (10 points) For the cycle \verb|2-1-5-3-2| what is the set $S$? 

{\blu
\[
S = \{1,2,3,5\}
\]
}			
			
			\item (20 points) What is the concrete constraint you would add to your model to properly eliminate the cycle \verb|2-1-5-3-2|? 
{
\blu

\[
x_{1,2}+x_{1,3}+x_{1,5}+x_{2,3}+x_{2,5}+x_{3,5} \leq 3
\]
}			
			
			\item (10 points) If you wanted to eliminate \textbf{all} cycles of size 4 from this graph, how many constraints would you have to write? 
			{
			\blu
			Have to eliminate $8$ choose $4$ cycles which is $70$
			}
			\end{enumerate}
		\end{enumerate}
\end{enumerate}

\end{document}
