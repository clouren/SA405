\documentclass[letterpaper,oneside,12pt]{article}%twocolumn;titlepage(separate title page)
\usepackage{pslatex}                %Times New Roman Font; Improves on-screen readability; preferred to using times package
\usepackage[margin=0.5in]{geometry}   %1in 'uniform' margin (with oneside)
\usepackage{color}
\usepackage{amsmath}
\usepackage{amsfonts}
\usepackage{enumitem}
\usepackage{url}
\usepackage{bm}
\usepackage{graphicx}
\usepackage{optprog}
\usepackage{lscape}
\setcounter{MaxMatrixCols}{20}
%\setlength{\headsep}{0in}           %Distance from bottom of header to the body of text on a page.
\newcommand{\rem}[1]{}
\renewcommand{\arraystretch}{1}

% **********************************************************************
% Compact itemize, enumerate and description environments %pkg enumitem
% **********************************************************************
\usepackage{enumitem}            % [FAILS w/BEAMER] Lets define enumerate spacing: \begin{enumerate}[topsep=0pt,itemsep=0pt,parsep=0pt,partopsep=0pt]
\newenvironment{compitem}{\begin{itemize}[topsep=0pt,itemsep=0pt,parsep=0pt,partopsep=0pt]}{\end{itemize}}
\newenvironment{compenum}{\begin{enumerate}[topsep=0pt,itemsep=0pt,parsep=0pt,partopsep=0pt]}{\end{enumerate}}
\newenvironment{compdesc}{\begin{description}[topsep=0pt,itemsep=0pt,parsep=0pt,partopsep=0pt]}{\end{description}}

\DeclareMathOperator{\conv}{conv}
\DeclareMathOperator*{\proj}{proj}

\newcommand{\blu}{\color{blue}}
\newcommand{\bla}{\color{black}}
\begin{document}
\noindent\fbox{%
	\parbox{\textwidth}{%
		\begin{center}
		\textbf{\large{Department of Mathematics}} \\
		\textbf{\large{SA 405 - Advanced Mathematical Programming}} \\
		\textbf{\large{Quiz 5}}
		\end{center}
	}%
}

\vspace{3mm} \hspace{\fill} \textbf{Name: \underline{\hspace{6cm}}}

\textbf{Part 1:}

You're going to Six Flags this weekend and are a roller coaster enthusiast. Six Flags has 7 total roller coasters. You label them 1 to 7 and plot out the distance you'd have to walk in order to get to each roller-coaster. The following matrix gives these distances (note it's symmetric):
\[
\begin{bmatrix}
-& 10 & 12 & 7  & 13 & 8  & 9 \\
 & -  & 10 & 13 & 25 & 6  & 7 \\
 &    & -  & 21 & 23 & 12 & 10 \\
 &    &    & -  & 12 & 3  & 9   \\
 &    &    &    & -  & 20 & 8   \\
 &    &    &    &    & -  & 3   \\
 &    &    &    &    &    & -
\end{bmatrix}
\]

You decide to model this problem as a Traveling Salesperson Problem so that you can visit each roller coaster while walking as little as possible. Suppose that once you solve it, your solver outputs the following information:

\verb|In order to minimize cost, the route is: 1-4-3-6-1 and 2-7-5-2|

\begin{enumerate}
\item (20 points) If the variables are $x_{i,j}$, what are the values of your variables corresponding to this solution? \vfill
\item (10 points)  What is the total cost of this solution? \vfill
\item (10 points) Is this an optimal solution? Is this a feasible solution? \vfill
\newpage
\item (20 points) If this solution is optimal, explain why. If it's not, write which constraints you would add to your model in order to exclude this solution. \vfill
\item (20 points) In TSP exactly two edges should be connected to each node. Write the constraint that will ensure that exactly two edges are connected  to node $1$. \vfill
\end{enumerate}

\textbf{Part 2:} Answer the following:

\begin{enumerate}[resume]
\item (10 points) True or false, when implementing the TSP problem in Python, one should implement every constraint in the model at the beginning and then solve it. \vspace{1in}
\item (10 points) Given a graph on $n$ nodes, how many edges can I choose in the graph without forming a cycle? \vspace{1in}
\end{enumerate}



\end{document}
