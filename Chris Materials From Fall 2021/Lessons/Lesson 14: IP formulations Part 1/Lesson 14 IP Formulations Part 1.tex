\documentclass[11pt]{article}

%% MinionPro fonts 
%\usepackage[lf]{MinionPro}
%\usepackage{MnSymbol}
\usepackage{microtype}

%% Margins
\usepackage{geometry}
\geometry{verbose,letterpaper,tmargin=1in,bmargin=1in,lmargin=1in,rmargin=1in}

%% Other packages
\usepackage{amsmath}
\usepackage{amsthm}
\usepackage[shortlabels]{enumitem}
\usepackage{titlesec}
\usepackage{soul}
\usepackage{tikz}
\usepackage{mathtools}
\usepackage{pgfplots}
\usepackage{tikz-3dplot}
\usepackage{algorithmic}
\usepackage[export]{adjustbox}
\usepackage{tcolorbox}
\usepackage{mathrsfs}
\usepackage{multicol}
\usepackage{framed}
\usepackage{optprog}

%% Paragraph style settings
\setlength{\parskip}{\medskipamount}
\setlength{\parindent}{0pt}

%% Change itemize bullets
\renewcommand{\labelitemi}{$\bullet$}
\renewcommand{\labelitemii}{$\circ$}
\renewcommand{\labelitemiii}{$\diamond$}
\renewcommand{\labelitemiv}{$\cdot$}

%% Colors
\definecolor{rred}{RGB}{204,0,0}
\definecolor{ggreen}{RGB}{0,145,0}
\definecolor{yyellow}{RGB}{255,185,0}
\definecolor{bblue}{rgb}{0.2,0.2,0.7}
\definecolor{ggray}{RGB}{190,190,190}
\definecolor{ppurple}{RGB}{160,32,240}
\definecolor{oorange}{RGB}{255,165,0}

%% Shrink section fonts
\titleformat*{\section}{\normalsize\bf}
\titleformat*{\subsection}{\normalsize\bf}
\titleformat*{\subsubsection}{\normalsize\it}

% %% Compress the spacing around section titles
\titlespacing*{\section}{0pt}{1.5ex}{0.75ex}
\titlespacing*{\subsection}{0pt}{1ex}{0.5ex}
\titlespacing*{\subsubsection}{0pt}{1ex}{0.5ex}

%% amsthm settings
\theoremstyle{definition}
\newtheorem{problem}{Problem}
\newtheorem{example}{Example}
\newtheorem*{theorem}{Theorem}
\newtheorem*{bigthm}{Big Theorem}
\newtheorem*{biggerthm}{Bigger Theorem}
\newtheorem*{bigcor1}{Big Corollary 1}
\newtheorem*{bigcor2}{Big Corollary 2}

%% tikz settings
\usetikzlibrary{calc}
\usetikzlibrary{patterns}
\usetikzlibrary{decorations}
\usepgfplotslibrary{polar}

%% algorithmic setup
\algsetup{linenodelimiter=}
\renewcommand{\algorithmiccomment}[1]{\quad// #1}
\renewcommand{\algorithmicrequire}{\emph{Input:}}
\renewcommand{\algorithmicensure}{\emph{Output:}}

%% Answer box macros
%% \answerbox{alignment}{width}{height}
\newcommand{\answerbox}[3]{%
  \fbox{%
    \begin{minipage}[#1]{#2}
      \hfill\vspace{#3}
    \end{minipage}
  }
}

%% \answerboxfull{alignment}{height}
\newcommand{\answerboxfull}[2]{%
  \answerbox{#1}{6.38in}{#2} 
}

%% \answerboxone{alignment}{height} -- for first-level bullet
\newcommand{\answerboxone}[2]{%
  \answerbox{#1}{6.0in}{#2} 
}

%% \answerboxtwo{alignment}{height} -- for second-level bullet
\newcommand{\answerboxtwo}[2]{%
  \answerbox{#1}{5.8in}{#2}
}

%% special boxes
\newcommand{\wordbox}{\answerbox{c}{1.2in}{.7cm}}
\newcommand{\catbox}{\answerbox{c}{.5in}{.7cm}}
\newcommand{\letterbox}{\answerbox{c}{.7cm}{.7cm}}

%% Miscellaneous macros
\newcommand{\tstack}[1]{\begin{multlined}[t] #1 \end{multlined}}
\newcommand{\cstack}[1]{\begin{multlined}[c] #1 \end{multlined}}
\newcommand{\ccite}[1]{\only<presentation>{{\scriptsize\color{gray} #1}}\only<article>{{\small [#1]}}}
\newcommand{\grad}{\nabla}
\newcommand{\ra}{\ensuremath{\rightarrow}~}
\newcommand{\maximize}{\text{maximize}}
\newcommand{\minimize}{\text{minimize}}
\newcommand{\subjectto}{\text{subject to}}
\newcommand{\trans}{\mathsf{T}}
\newcommand{\bb}{\mathbf{b}}
\newcommand{\bx}{\mathbf{x}}
\newcommand{\bc}{\mathbf{c}}
\newcommand{\bd}{\mathbf{d}}

%% LP format
%    \begin{align*}
%      \maximize \quad & \mathbf{c}^{\trans} \mathbf{x}\\
%      \subjectto \quad & A \mathbf{x} = \mathbf{b}\\
%                       & \mathbf{x} \ge \mathbf{0}
%    \end{align*}

%Space between rows:
%\def\arraystretch{2.2}
%
%Space between columns:
%\arraycolsep=1.4pt


%% Redefine maketitle
\makeatletter
\renewcommand{\maketitle}{
  \noindent SA405 -- AMP \hfill Rader \S 13.1  \\

  \begin{center}\Large{\textbf{\@title}}\end{center}
}
\makeatother

%% ----- Begin document ----- %%
\begin{document}
  
\title{Lesson 14.  IP Formulations}

\maketitle

%%%

\renewcommand\labelitemi{--}
\section{Solving Integer Programs can be \emph{Really} Hard!}

The following integer (linear) program (IP) seeks an objective-maximizing integer linear combination of a big number. 
%% LP format
\begin{align*}
      \maximize \quad & 213x_1 - 1928x_2 - 11111x_3 - 2345x_4 + 9123x_5
\\
      \subjectto \quad & 12223x_1 + 12224x_2 + 36674x_3 + 61119x_4 + 85569x_5 = 89643482\\
                       & x_1,~x_2,~x_3,~x_4,~x_5 \ge 0, \text{ integer }
\end{align*}

If we implement this problem in python (using GLPK), and solve it as a \textbf{linear program} we obtain the following solution relatively fast:
\verb|(0,0,0,0,1047.62)|.

If you solve it as an integer program (using GLPK) it can take \textbf{hours} to solve!

\begin{itemize}
\item Only has 5 variables
\item Problems like this are the ones I look at in my research :)
\end{itemize}

\begin{tcolorbox}
In general, IPs are \textbf{significantly harder} to solve than LPs.
\end{tcolorbox}

\begin{itemize}
\item In the next two lessons, we will discuss why IPs are harder than LPs and why the way we model IP problems can impact solver performance.
\item In lesson 16 we will learn about the \textbf{branch and bound algorithm} which can be used to solve IPs.
\end{itemize}

\newpage
\section{Review Linear Programming Solution Techniques}

\subsection{Types of LP solutions}

\textbf{Theorem:} Every LP's solution is EXACTLY one of the following:
	\begin{enumerate}
	\item Unique optimal solution
	\item Multiple optimal solutions
	\item Unbounded LP
	\item Infeasible LP
	\end{enumerate}

\begin{problem}
Sketch graphs which illustrate each of the types of LP solutions.
\end{problem}

\vfill

\begin{tcolorbox}
These types of solutions are also true for integer programs
\end{tcolorbox}

\newpage

LPs are solved via the simplex method. Small LPs can be solved graphically.

\begin{problem}  
Review: Solve the following LP graphically:
\begin{optprog*}
\text{max} & \objective{x_1+x_2} \\
st & x_1+2x_2 & \leq & 10 \\
  & x_1 & \leq & 4 \\
  & x_1, x_2 & \geq & 0
\end{optprog*}

\end{problem}

\vspace{3in}

\begin{tcolorbox}
\textbf{Theorem:} If an LP has an optimal solution (i.e., it is not unbounded or infeasible), the optimal solution occurs at a \textbf{corner point} of the feasible region.
\end{tcolorbox}

Question: Is the same true for an IP?

\section{Integer Program Formulations}

\begin{tcolorbox}
A \textbf{formulation} of an integer (linear) program is a set of linear \wordbox that capture ALL of the \wordbox integer points, and NO OTHER integer points.
\end{tcolorbox}

\vfill

\begin{tcolorbox}
The \textbf{LP relaxation} of an IP is the LP that is formed by relaxing (i.e., removing) the integer requirement on the variables.
\end{tcolorbox}

\newpage
\begin{problem}  
Below are two integer programs, along with the diagrams of their constraints.

\begin{multicols}{2}

{\bf Integer Program A}
\begin{align*}
      \maximize \quad & 8x + 7y \\
      \subjectto \quad & -18x + 38y ~~\leq~~ 133\\
                       & 13x + 11y ~\leq~ 125\\
                       & 10x - 8y ~\leq~ 55\\
                       & x,~y ~\geq~ 0, \text{ integer}
\end{align*}
\vspace{6cm}

\includegraphics[width = 0.4\textwidth]{formulation_bad}
\end{multicols}

\begin{multicols}{2}

{\bf Integer Program B}
\begin{align*}
      \maximize \quad & 8x + 7y \\
      \subjectto \quad & -x+2y ~\leq~ 6\\
                       & x + y ~\leq~ 10\\
                       & x - y ~\leq~ 5\\
                       & x ~\leq~ 7\\
                       & y ~\leq~ 5\\
                       & x,~y ~\geq~ 0, \text{ integer}
\end{align*}
\vspace{6cm}

\includegraphics[width = 0.4\textwidth]{formulation_ideal}
\end{multicols}


\begin{enumerate}[(a)]
   \item On the diagrams, identify all feasible solutions to both IPs.
   \item Are the integer feasible regions for IP A and IP B different or the same? \vspace{0.5in}
   \item Are the feasible regions of the LP relaxations of IP A and IP B different or the same? \vspace{0.5in}
   \item What does this mean about problems A and B? \vspace{0.5in}
   \item Based on these graphs, will the optimal solution of an IP always occur at a corner point? \vspace{0.5in}
   \item Which of these formulations is easier to solve? Why? \vspace{1in}

\end{enumerate}
\end{problem}

\subsection{Better Formulation $\Rightarrow$ Better Bound}

Now let's consider the relationship between an IP and its LP relaxation.

\bigskip

In general:
	\begin{itemize}
	\item If we are solving a \textbf{maximization} IP, the solution of its LP relaxation provides a \wordbox bound on the solution of the IP problem.
	\item If we are solving a \textbf{minimization} IP, the solution of its LP relaxation provides a \wordbox bound on the solution of the IP problem.
	\end{itemize}

The \textbf{tighter} a formulation, the \wordbox bound you obtain via the LP relaxation.

\begin{tcolorbox}
This idea is key for solving IPs!
\end{tcolorbox}	
\begin{problem}
Sketch a problem which proves if we're maximizing, $z_{LP} \geq z_{IP}$ and vice versa if minimizing.
\end{problem}

Often the decision of how to formulate an IP comes down to a tradeoff between the formulation quality and number of constraints.
\begin{itemize}
\item More constraints can lead to a better (tighter) formulation, but: \vspace{0.5in}
\item Fewer constraints lead to a weaker formulation but:
\end{itemize}	





\end{document}
